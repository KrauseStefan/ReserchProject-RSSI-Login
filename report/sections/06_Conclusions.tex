\section{Conclusion}
Using RSSI value as viable means for proximity detection is possible but inaccurate. Therefore we implemented a solution which uses a filter and a hysteresis threshhold combined, to remove noise from RSSI measurements and optimize accuracy and thereby make a more consistent proximity detection.

RSSI values depends on multiple factors, such as distance, reflections from environment, transmission strength, etc., which is reflected in our test results. Environment knowledge prior to implementing authentication using RSSI values is therefore crucial to be able to calibrate the base station correctly.

We have developed an application that is capable of detecting a device, and based on the RSSI value off the bluetooth signal of the device to authenticate or deauthenticate a user. Because of the variable nature of RSSI the applications tresholds must be calibrated to the surrounding environment in order to make the authentication procedure work correct. Source code is available on github \url{https://github.com/KrauseStefan/ReserchProject-RSSI-Login/tree/master/testApp}.

%
%A partial authentication model has been proposed taking the security aspect of calm authentication into consideration. The model has three levels of trust: Low, Medium and High allowing different read/write/edit rights according to the level of trust.

%We have shown that context aware authentication using Bluetooth Low Energy is possible using RSSI measurements for proximity detection.